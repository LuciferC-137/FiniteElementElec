\documentclass{article}

\usepackage{biblatex}
\addbibresource{biblio.bib}
\usepackage{graphicx}
\usepackage[utf8]{inputenc}
\usepackage[margin=1.2in]{geometry}
\usepackage{color}
\usepackage{amsmath}
\usepackage{hyperref}
\usepackage{listings}
\usepackage{xcolor}

\definecolor{methods_yellow}{rgb}{0.7, 0.6, 0.0}
\definecolor{variable_bleu}{rgb}{0.3, 0.6, 0.8}
\definecolor{keywords_purple}{rgb}{0.7, 0.0, 0.6}
\definecolor{comments_green}{rgb}{0.1, 0.5, 0.1}
\definecolor{listing_background}{rgb}{0.15, 0.15, 0.15}
\definecolor{dark_white}{rgb}{0.9, 0.9, 0.9}

\lstset{
  language=Python,
  keywordstyle=\color{keywords_purple},
  identifierstyle=\color{variable_bleu},
  emph={solve_mesh,compute_element_stiffness_matrix,
        apply_boundary_conditions,compute_rigidity_matrix},
  emphstyle=\color{methods_yellow},
  commentstyle=\color{comments_green},
  backgroundcolor=\color{listing_background},
  basicstyle=\ttfamily\small\color{dark_white},
  xleftmargin=1em,
  framexleftmargin=0.5em,
  framexrightmargin=0.5em,
  frame=single,
  breaklines=true,
  tabsize=2
}

\begin{document}

\begin{titlepage}
    \begin{titlepage}      
        \begin{center}
            \includegraphics[width=7cm]{img/Logo_IMT_Atlantique.png}
            \\[1.9cm]
            {\LARGE IMT Atlantique}\\[1.5cm]
            
			{\color{black} \rule{\textwidth}{1pt}}

            \linespread{1.2}
            \vspace{0.5cm}
            \huge {\textbf{Électrostatique : modélisation de l'effet de pointe
             par méthode des éléments finis}}
            \vspace{1cm}
            \linespread{1.2}
			{\color{black} \rule{\textwidth}{1pt}}
            \Large {BLOCH Maxime} \\
            \Large {LEHEL Eliaz} \\[1cm]
            \Large {Mars 2025}
            
        \end{center}
    \end{titlepage}
\end{titlepage}

\renewcommand{\contentsname}{Sommaire}
\tableofcontents

\newpage

\section*{Introduction}

\hspace{1cm}
Nous nous proposons dans ce travail d'utiliser
la Méthode des Eléments Finis (MEF)
pour modéliser l'effet de pointe dans un champ électrostatique.\cite{Gossiaux}

\section{Situation physique}
\label{sec:situation_physique}

\begin{figure}[!h]
    \centering
    \includegraphics[scale=0.7]{img/schema.png}
    \caption{Modélisation de la pointe}
    \label{fig:pointe}
\end{figure}

\hspace{0.5cm}
La figure \ref{fig:pointe} représente une pointe (en gris) dont les bords sont
maintenus à un potentiel $V=0$ et les contours de la cavité (en noir) à
un potentiel constant $V(\theta)$ non nul.

\begin{equation}
    V(\theta) = 1 - \frac{\theta^2}{\theta_0^2},
    \ \text{avec } \theta_0=\frac{3\pi}{4}
\end{equation}

Dans ces conditions, le potentiel électrostatique $V$ est solution de l'équation
de Laplace :

\begin{equation}
    \Delta V = 0
\end{equation}

\newpage

\section{Modélisation par éléments finis}

\subsection{Equation à résoudre}

\hspace{0.5cm}
Nous cherchons à résoudre l'équation de Laplace sur un domaine $\Omega$ avec des
conditions aux limites de Dirichlet sur le domaine $\Gamma_D$.

On pose $u$ le potentiel électrostatique. Nous voulons résoudre:

\begin{equation}
   \frac{\partial^2 u}{\partial x^2} + \frac{\partial^2 u}{\partial y^2} = 0
\end{equation}

Nous introduisons les fonctions $v$ sur $\Omega$ le domaine de définition de $u$,
notre potentiel, avec $v \in V = \{H^1(\Omega); v|_{\Gamma_D} = 0 \}$, où
$H^1(\Omega)$ est l'espace de Hilbert des fonctions dérivables au sens des
distributions et carrés intégrables sur $\Omega$.

Pour appliquer la méthode des éléments finis, nous utilisons Green
pour écrire:

\begin{equation}
    \int_\Omega \Delta u(x) v(x) dx = \int_\Omega \nabla u(x) \nabla v(x) dx
    - \int_{\Gamma} \partial_n u(s) v(s) ds
\end{equation}

Par définition, les fonctions tests $v$ sont nulles sur le bord $\Gamma$. Nous avons
donc:

\begin{equation}
    \int_\Omega \nabla u(x) \nabla v(x) dx = 0
    \label{eq:weak_form}
\end{equation}

L'équation \ref{eq:weak_form} est appelée la formulation
faible de notre problème.

Ainsi, nous cherchons $u$ telle que pour tout $v$ dans l'espace
de Hilbert $H^1(\Omega)$, l'équation \ref{eq:weak_form} soit vérifiée.

On peut réécrire cette équation sous la forme suivante:

\begin{equation}
    a(u, v) = 0
\end{equation}

On ajoute maintenant la condition de Dirichlet sur le bord $\Gamma_D$:

\begin{equation}
    u(x) = g(x), \hspace{0.2cm} \forall x \in \Gamma_D
    \Leftrightarrow  u-g \in V
\end{equation}

Dès lors, nous pouvons approcher le problème avec la solution 
$u^h$ dans un espace de dimension finie $V^h$.

On a donc:

\begin{equation}
    a(u^h, v^h) = 0
\end{equation}

On décompose $u^h$ et $v^h$ sur la base des fonctions chapeau
$\{B^h_i\}^{i=N^h}_{i=1}$

\begin{equation}
    u^h = g^h + \sum_{i=1}^{N^h} x_i B^h_i
    \ \text{ et } \ v^h = \sum_{i=1}^{N^h} y_i B^h_i
\end{equation}

Avec $x_i$ les inconnues du problème.

On revient au problème équivalent suivant:

\begin{equation}
    a(u^h - g^h, v^h) = - a(g^h, v^h)
\end{equation}

On pose maintenant les coefficients $A_{ij} = a(B^h_j, B^h_i)$. Ainsi, nous pouvons
réécrire le problème sous forme matricielle:

\begin{equation}
    \sum_{i,j} y_i A_{i,j} x_j
\end{equation}

En posant $g_i = a(g^h, B^h_i)$, ce qui donne $a(g^h, v^h) = \sum_i y_i g_i$,
on obtient le système linéaire suivant:

\begin{equation}
    \sum_{i,j} y_i A_{i,j} x_j = - \sum_i y_i g_i
\end{equation}

Que l'on peut réécrire sous forme matricielle en posant $b$ le vecteur colonne
des $-g_i$ et $x$ le vecteur colonne des $x_i$:

\begin{equation}
    y^T \cdot A \cdot x =  y^T \cdot b
\end{equation}

Ce qui équivaut à résoudre:

\begin{equation}
    A \cdot x = b
\end{equation}

\subsection{Méthode des éléments finis}

Nous devons maintenant assurer le raccord entre les différents éléments.
Pour ce faire, nous considérons dorénavant les fonctions tests
polynomiales $v \in X^h$.

Dès lors, on montre que chaque fonction $v$
est définie par les valeurs qu'elle prend
sur les nœuds du maillage et nous pouvons les écrire comme une combinaison
linéaire des fonctions $B_i$, les fonctions chapeau généralisées.

Ainsi, nous assurons directement la continuité de $v$ sur les éléments en
ayant simplement les mêmes valeurs aux nœuds.

\newpage

\subsection{Maillage du domaine}
\label{sec:maillage}

\hspace{0.5cm}
Nous allons discrétiser le domaine en éléments finis et résoudre le problème
de Laplace par la méthode des éléments finis. Dans un premier temps,
nous avons choisi de considérer une géométrie simple, mais les calculs
qui vont suivre demeurent généraux jusqu'à la section \ref{sec:cadre_du_pb}.

La figure \ref{fig:pointe_ef} représente la discrétisation du domaine en éléments
finis. Nous nous placerons en deux dimensions.

\begin{figure}[h]
    \centering
    \hspace{1cm} \includegraphics[scale= 0.7]{img/pointe_ef.png}
    \caption{Discretisation du domaine}
    \label{fig:pointe_ef}
\end{figure}

On rappelle qu'on doit résoudre le problème suivant:

\begin{equation}
    \begin{cases}
        u^h \in X^h, \hspace{0.3cm} u^h - g^h \in V^h \\
        a(u^h, v^h) = 0, \hspace{0.3cm} \forall v^h \in V^h
    \end{cases}   
\end{equation}

\newpage

\subsection{Décomposition des polynômes sur un élément}

On réécrit le problème avec des sommes sur chaque élément $[e]$:

\begin{equation}
    a(u^h, v^h) = \sum_{T^{[e]} \in \mathcal{T}^h} \int_{T^{[e]}}
    \sum_{i,j=1}^N a_{ij} \partial_j u ^{[e]} \partial_i v^{[e]} dT^{[e]}
    \label{eq:general_problem}
\end{equation}

Chaque élément $T^{[e]}$ est défini par trois
nœuds (Figure \ref{fig:element}).

\begin{figure}[h]
    \centering
    \includegraphics[scale=0.5]{img/element.png}
    \caption{Un élément du maillage}
    \label{fig:element}
\end{figure}

Dans la suite, l'idée sera de décomposer les polynômes
sur chaque élément et de réécrire le problème sous forme
matricielle.

Pour ce faire, on note simplement $u^{[e]}$ et $v^{[e]}$
les fonctions $u$ et $v$ sur l'élément $T^{[e]}$, ainsi que
$u^{[e]}_i$ et $v^{[e]}_i$ les valeurs de $u^{[e]}$ et $v^{[e]}$
au nœud $i$ de l'élément $T^{[e]}$.

\subsection{Matrice de raideur partielle}

On introduit la matrice de raideur partielle
$K^{[e]}$ telle que pour un élément [e]:

\begin{equation}
    \int_{T^{[e]}} \sum_{i,j=1}^2 a_{ij} \partial_j u ^{[e]}
     \partial_i v^{[e]} dT^{[e]} 
     = \left[ v_1^{[e]}, v_2^{[e]},  v_3^{[e]} \right] \cdot K^{[e]}
     \cdot \left[ u_1^{[e]}, u_2^{[e]}, u_3^{[e]} \right]^T
    \label{eq:int_to_K}
\end{equation}

Si nous considérons un seul élément [e], nous pouvons chercher à expliciter
les termes de la matrice de raideur $K^{[e]}$.

On réécrit le problème comme suit:

\begin{equation}
    \left[ v_1^{[e]}, v_2^{[e]},  v_3^{[e]} \right] K^{[e]}
     \left[ u_1^{[e]}, u_2^{[e]}, u_3^{[e]} \right]^T
     = \int_{T^{[e]}} \left[\partial_x v^{[e]},
     \partial_y v^{[e]}\right] A^{[e]} \left[\partial_x u^{[e]},
     \partial_y u^{[e]}\right]^T dT^{[e]}
\end{equation}

Avec $A^{[e]}$ la matrice des coefficients $a_{ij}$.

Nous allons chercher à réécrire ce problème sous forme matricielle
afin d'être capable de calculer les coefficients de $K^{[e]}$.
De cette manière, nous serons capable d'avoir accès au terme
de l'intégrale de l'équation \ref{eq:int_to_K}.

En sélectionnant des polynômes de degré 1 sur un élément $T^{[e]}$,
l'intégrale se réduit à la mesure de l'aire du triangle élémentaire
noté $|T^{[e]}|$ et à moyenner les valeurs de $A^{[e]}$:

\begin{equation}
    \begin{bmatrix}
        v_1^{[e]} & v_2^{[e]} &  v_3^{[e]}
    \end{bmatrix}
    K^{[e]}
    \begin{bmatrix}
        u_1^{[e]} \\ u_2^{[e]} \\ u_3^{[e]}
    \end{bmatrix}
     = |T^{[e]}| \cdot
    \begin{bmatrix}
        \partial_x v^{[e]} & \partial_y v^{[e]}
    \end{bmatrix}
    \overline{A^{[e]}}
    \begin{bmatrix}
        \partial_x u^{[e]} \\ \partial_y u^{[e]}
    \end{bmatrix}
    \label{eq:vKu_eq}
\end{equation}

\subsection{Réécriture sous forme matricielle}

Puisque $u,v \in V^h$, on sait qu'on peut les écrire sous forme
de polynômes de degré 1 en deux dimensions:

\begin{align}
    u^{[e]}(x,y) = \alpha_0^{[e]} + \alpha_1^{[e]} x + \alpha_2^{[e]} y
    = \left[1 \ x \ y\right]
    \left[ \alpha_0^{[e]} \ \alpha_1^{[e]} \ \alpha_2^{[e]} \right]^T \\
    v^{[e]}(x,y) = \beta_0^{[e]} + \beta_1^{[e]} x + \beta_2^{[e]} y
    = \left[1 \ x \ y\right]
    \left[ \beta_0^{[e]} \ \beta_1^{[e]} \ \beta_2^{[e]} \right]^T
\end{align}

Puisqu'on cherche la décomposition de $u$ sur la 
base des fonctions chapeau, on exprime les coefficients de $u$
par rapport à cette base grâce aux sommets de l'élément
sous la forme suivante:

\vspace{0.5cm}

\begin{minipage}{0.45\textwidth}
    \begin{equation*}\begin{cases}
        \alpha_0^{[e]} + \alpha_1^{[e]} x_1
        + \alpha_2^{[e]} y_1 = u^{[e]}_1 \\
        \alpha_0^{[e]} + \alpha_1^{[e]} x_2
        + \alpha_2^{[e]} y_2 = u^{[e]}_2 \\
        \alpha_0^{[e]} + \alpha_1^{[e]} x_3
        + \alpha_2^{[e]} y_3 = u^{[e]}_3
    \end{cases}\end{equation*}
\end{minipage}
\begin{minipage}{0.01\textwidth}
    \centering
    $\Leftrightarrow$
\end{minipage}
\begin{minipage}{0.45\textwidth}
    
    \begin{equation}
        \hspace{-2cm} P^{[e]} \left[\alpha^{[e]}\right] = \left[u^{[e]}\right]
        \label{eq:p_alpha_u}
    \end{equation}
\end{minipage}

\vspace{5mm}

Avec la matrice $P^{[e]}$ définie par:

\begin{equation}
    P^{[e]} = \begin{bmatrix}
        1 & x_1 & y_1 \\
        1 & x_2 & y_2 \\
        1 & x_3 & y_3
    \end{bmatrix}
\end{equation}

Par ailleurs, en dérivant les formes générales
de $u$ et $v$, nous pouvons écrire:

\begin{align}
    \begin{bmatrix}
    \partial_x u^{[e]} \\ \partial_y u^{[e]}
    \end{bmatrix}
    =
    \begin{bmatrix}
    0 & 1 & 0 \\
    0 & 0 & 1
    \end{bmatrix}
    \begin{bmatrix}
    \alpha_0^{[e]} \\ \alpha_1^{[e]} \\ \alpha_2^{[e]}
    \end{bmatrix} \\
    \begin{bmatrix}
    \partial_x v^{[e]} \\ \partial_y v^{[e]}
    \end{bmatrix}
    =
    \begin{bmatrix}
    0 & 1 & 0 \\
    0 & 0 & 1
    \end{bmatrix}
    \begin{bmatrix}
    \beta_0^{[e]} \\ \beta_1^{[e]} \\ \beta_2^{[e]}
    \end{bmatrix}
\end{align}

On pose alors la matrice $D =
\begin{bmatrix}
    0 & 1 & 0 \\
    0 & 0 & 1
\end{bmatrix} $
qui nous permet de dériver simplement les fonctions $u$ et $v$.
On pose les vecteurs suivants:

\begin{equation}
    \begin{bmatrix}
        \partial_x u^{[e]} \\ \partial_y u^{[e]}
    \end{bmatrix}
    =
    \left[ \partial u^{[e]} \right]
    \text{ et }
    \begin{bmatrix}
        \partial_x v^{[e]} \\ \partial_y v^{[e]}
    \end{bmatrix}
    =
    \left[ \partial v^{[e]} \right]
\end{equation}

\begin{equation}
    \begin{bmatrix}
        \beta_0^{[e]} \\ \beta_1^{[e]} \\ \beta_2^{[e]}
    \end{bmatrix}
    = \left[ \beta^{[e]} \right]
    \text{ et }
    \begin{bmatrix}
        \alpha_0^{[e]} \\ \alpha_1^{[e]} \\ \alpha_2^{[e]}
    \end{bmatrix}
    = \left[ \alpha^{[e]} \right]
\end{equation}

\begin{equation}
    \begin{bmatrix}
        u_1^{[e]} \\ u_2^{[e]} \\ u_3^{[e]}
    \end{bmatrix}
    = \left[ u^{[e]} \right]
    \text{ et }
    \begin{bmatrix}
        v_1^{[e]} \\ v_2^{[e]} \\ v_3^{[e]}
    \end{bmatrix}
    = \left[ v^{[e]} \right]
\end{equation}

Avec ces nouvelles matrices, muni de l'équation
\ref{eq:p_alpha_u} et de l'équation \ref{eq:vKu_eq} que l'on réécrit
comme suit:

\begin{equation}
    \left[ v^{[e]} \right]^T K^{[e]} \left[ u^{[e]} \right]
    =
    |T^{[e]}| \left[ \partial v^{[e]} \right]
    \overline{A^{[e]}} \left[ \partial u^{[e]} \right]
\end{equation}

On peut alors réécrire le problème sous forme matricielle:

\vspace{0.4cm}

\begin{equation}
    \left[ v \right]^T K^{[e]} \left[ u \right]
    =
    |T^{[e]}| \left[ \partial v \right]^T \overline{A^{[e]}}
    \left[ \partial u \right]
\end{equation}

\vspace{-0.3cm}

\begin{center} $\Leftrightarrow$ \end{center}

\vspace{-0.6cm}

\begin{equation}
    \left(P^{[e]}\left[\beta^{[e]}\right] \right)^T K^{[e]}
    P^{[e]}\left[\alpha^{[e]}\right] = |T^{[e]}|
    \left(D\left[\beta^{[e]}\right]\right)^T
    \overline{A^{[e]}} D\left[\alpha^{[e]}\right]
\end{equation}

\vspace{0.4cm}

En passant les termes à droite, en simplifiant les
matrices $\left[\alpha\right]$ et $\left[\beta\right]$ par
la droite et la gauche, et en écrivant
$H^{[e]} = \left( P^{[e]}\right)^{-1}$, on obtient:

\begin{equation}
    K^{[e]} = \left(H^{[e]}\right)^T D^T
    \left( |T| \overline{A^{[e]}} \right) D H^{[e]}
\end{equation}

Notez qu'ici, $D$ est de taille 2x3, $\overline{A^{[e]}}$
de taille 2x2 et $H^{[e]}$ de taille 3x3.

\subsection{Expression des matrices dans le cadre du problème}
\label{sec:cadre_du_pb}

En pratique, avec notre problème, nous savons que:

\begin{equation}
    \overline{A^{[e]}} =
    \begin{bmatrix}
        1 & 0 \\ 0 & 1
    \end{bmatrix}
\end{equation}

Car, rappelons-le, notre formulation faible d'origine est

\begin{equation}
    \int_\Omega \nabla u(x) \nabla v(x) dx = 0
\end{equation}

Nous avons maintenant besoin de connaître la matrice $H^{[e]}$
et la valeur de $|T|$.

Notons $h$ la longueur d'un côté de l'élément [e]. Nous avons donc
$h = x_2 - x_1 = y_2 - y_1$. Dès lors,

\begin{equation}
    |T| = \frac{1}{2} h^2
\end{equation}


Par ailleurs, nous pouvons inverser la matrice $P^{[e]}$ pour obtenir
la matrice $H^{[e]}$. Par calcul, on trouve:

\begin{equation}
    H^{[e]} = \frac{1}{h^2}
    \begin{bmatrix}
        x_2 y_2 - x_1 y_1 & -hx_1 & -hy_1 \\
        -h & h & 0 \\
        -h & 0 & h
    \end{bmatrix}
\end{equation}

Ces simplifications pourraient être faites dans notre code pour
accélérer les calculs, mais en pratique, on constate que le temps de
calcul est principalement occupé par la reconstruction des valeurs
en chaque point du plan (x, y) (pour une certaine résolution donnée)
plutôt que par la résolution elle-même.

Nous avons donc choisi de garder dans notre solveur une forme générale
du problème telle que présentée à la section précédente.
(La forme de la matrice $A^{[e]}$ est cependant conservée égale à l'identité,
car cette matrice dépend du problème physique et non de la géométrie.)

Ceci nous permet, comme nous le verrons dans la suite,
d'adopter au besoin de nouvelles géométries.

\subsection{Assemblage}

Maintenant dotés des matrices de rigité élémentaires $K^{[e]}$
sur chaque élément, l'équation \ref{eq:general_problem} nous permet
de connaitre les coefficients de $K$, la matrice de rigidité globale
du problème.

En effet, nous avons la correspondance suivante:

\begin{equation}
    \begin{bmatrix}
        v^h_1 & \cdots & v^h_N
    \end{bmatrix}
    K
    \begin{bmatrix}
        u^h_1 \\ \vdots \\ u^h_N
    \end{bmatrix}
    = \sum_{T^{[e]} \in \mathcal{T}^h}
    \begin{bmatrix}
        v^{[e]}_1 & v^{[e]}_2 & v^{[e]}_3
    \end{bmatrix}
    K^{[e]}
    \begin{bmatrix}
        u^{[e]}_1 \\ u^{[e]}_2 \\ u^{[e]}_3
    \end{bmatrix}
    \label{eq:assemblage}
\end{equation}

Ce que nous dit cette équation, c'est que pour chaque nœud,
la valeur de K est la somme de toutes les valeurs de $K^{[e]}$
correspondantes à ce nœud. En effet, chaque élément [e] est
défini par trois nœuds, et chaque nœud est partagé par trois
éléments, donc une matrice $K^{[e]}$ contient
plusieurs fois les contributions de chaque nœud.

\subsection{Conditions limites}

Il ne reste plus qu'à définir les conditions limites de notre système.

Si aucune contrainte n'existait, le système à résoudre serait:

\begin{equation}
    K \left[u^h\right]^T = \left[0\right]
\end{equation}

Mais certaines contributions doivent être retirées de la matrice K.
En effet, il ne faut pas calculer les variations du nœud $i$ lorsque
celui-ci est sur le bord, donc les contributions des autres nœud
à la valeur en $i$ doivent être éliminées. Pour ce faire, il suffit
de fixer à $0$ la colonne $i$ de la matrice $K$
(sauf la valeur en $(i, i)$, ainsi le nœud $i$ sera le seul à contribuer
à sa propre valeur, et on élimine la variation) et définir une valeur
fixe dans le second membre à la place de $0$.

Admettons dans l'exemple suivant que le nœud $i$ se trouve au bord et
est fixée à la valeur $U_i$
Voici les modifications à apporter à la matrice:

\begin{equation}
    \begin{bmatrix}
        K_{1,1} & \cdots & K_{i,1} = 0 & \cdots & K_{1,N} \\
        \vdots & \ddots & \vdots & \ddots & \vdots \\
        \vdots & \ddots & K_{i,i} = 1 & \ddots & \vdots \\
        \vdots & \ddots & \vdots & \ddots & \vdots \\
        K_{N,1} & \cdots & K_{i,N}=0 & \cdots & K_{N,N}
    \end{bmatrix}
    \begin{bmatrix}
        u^h_1 \\ \vdots \\ u^h_N
    \end{bmatrix}
    =
    \begin{bmatrix}
        0 \\ \vdots \\ U_i \\ \vdots \\ 0
    \end{bmatrix}
\end{equation}

En notant le membre de droite $F$ et $K^*$ la nouvelle
matrice de rigidité prenant en compte les conditions aux limites,
il en découle simplement un système linéaire à résoudre:

\begin{equation}
    K^*
    \begin{bmatrix}
        u^h_1 \\ \vdots \\ u^h_N
    \end{bmatrix}
    = F
\end{equation}

Et c'est bien ce calcul qui sera réalisé dans notre code.

\newpage

\section{Présentation du code}

L'architecture du code, la présentation des classes et de leurs
méthodes est déjà décrites extensivement dans le \verb|README|
du projet ainsi que sur le wiki associée.
(\href{https://github.com/LuciferC-137/FiniteElementElec}{Lien vers GitHub})

\subsection{Solveur}

Dans cette section est détaillé le fonctionnement
mathématique du solveur et comment il implémente les équations
présentées précédemment.

Il y a quatre méthodes principales dans le solveur:

\begin{itemize}
    \item \verb|solve_mesh()|
    \item \verb|compute_rigidity_matrix()|
    \item \verb|apply_boundary_conditions()|
    \item \verb|compute_element_stiffness_matrix()|
\end{itemize}

La première résout le problème en appelant les autres.
La deuxième calcule la matrice de raideur $K$ nécessaire à la
résolution du problème, la troisième applique les conditions
limites à la matrice de raideur pour obtenir $K^*$ et $F$, et la dernière calcule
une matrice élémentaire $K^{[e]}$.

Bien évidemment, elles ne sont pas appelées dans cet ordre. Voici
comment se déroule l'algorithme:

\begin{enumerate}
    \item Pour chaque éléments du maillage,
    on calcule la matrice de raideur élémentaire $K^{[e]}$.
    \item Pour chaque noeud, on ajoute ses contributions en "ajoutant" $K^{[e]}$ à l'endroit approprié de $K$.
    \item On applique les conditions limites à $K$ pour obtenir $K^*$ et on détermine $F$.
    \item On résout le système linéaire $K^* \cdot u = F$.
\end{enumerate}

Une version de pseudo-code simplifiée et verbalisée de chaque méthode 
est présentée au listing \ref{lst:solver} ci-après.

\newpage

\begin{lstlisting}[caption={Structure simplifiée du solveur} \label{lst:solver}]
    def solve_mesh():
        # On cree la matrice avec toutes les contributions
        K = compute_rigidity_matrix()
        F = np.zeros(mesh.size())
        # On s'assure de retirer les contributions des noeuds a valeur fixe
        apply_boundary_conditions(K, F, mesh)
        # On appel numpy pour resoudre le systeme lineaire
        u = np.linalg.solve(K, F)
        return u
    
    def compute_rigidity_matrix():
        K = np.zeros((n_nodes, n_nodes))
        for element in mesh.elements():
            # Pour chaque element, on rempli Ke
            Ke = compute_element_stiffness_matrix(element)
            for i, node_i in element.nodes:
                for j, node_j in element.nodes:
                    # On ajoute le coeff de Ke correspondant a la
                    # contribution entre deux noeuds a la matrice K
                    K[node_i.index, node_j.index] += Ke[i, j]
        return K
    
    def compute_element_stiffness_matrix(element):
        x = [node.x for node in element]
        y = [node.y for node in element]

        Pe = np.array([[1, x[0], y[0]],
                       [1, x[1], y[1]],
                       [1, x[2], y[2]]])
        Ae = np.array([[1, 0], [0, 1]])
        He = np.linalg.inv(Pe) # Inversement de la matrice Pe
        Te = 0.5 * np.abs(np.linalg.det(Pe)) # Calcul de l'aire du triangle
        D = np.array([[0, 1, 0],
                      [0, 0, 1]]) # Matrice de derivation
        DT = np.array([[0, 0],
                       [1, 0],
                       [0, 1]]) # Transpose de D

        Ke = np.transpose(He) @ DT @ Ae @ D @ He * Te # Solution elementaire
        return Ke

    def apply_boundary_conditions(K):
        for i in range(mesh.size()):
            if mesh[i].value is not None:
                # Si on a une condition limite au noeud i
                K[i, :] = 0
                # Alors on retire toutes les contributions des autres noeuds
                K[i, i] = 1
                # On devient le seul noeud a participer a notre valeur
                F[i] = mesh[i].value
                # Et la solution pour ce noeud sera nous-meme
    
    \end{lstlisting}

\newpage

\subsection{Maillage disponibles et architecture}

Le code a été pensé selon une programmation orientée objet
avec le langage de programmation \verb|Python|.

Cette architecture a l'avantage de permettre une configuration
simple de nouveaux maillages, et c'est la raison pour laquelle
nous avons proposé deux options: un maillage carré comme présenté
dans la section \ref{sec:maillage}, et un maillage circulaire
plus proche de la situation étudiée d'après le schéma
figure \ref{fig:pointe}.

Chacun de ces maillages implémente une classe abstraite
\verb|Mesh| qui définit les propriétés que doit avoir
le maillage afin d'être traité par le solveur.

Puisque le solveur est codé de manière générale, un utilisateur
pourrait définir son propre maillage grâce aux classe de notre
code et le résoudre simplement.

Nous faisons également remarquer que la création d'un maillage
circulaire n'est pas optimisée et peut rapidement prendre du temps.
Cependant, le solveur est plutôt rapide grâce à l'utilisation
du package \verb|numpy| qui permet une résolution optimisée
de systèmes matriciels.


\subsection{Forme des résultats}

Le code offre la possibilité d'afficher de nombreux graphiques
différents, que ce soit pour assister la construction de maillage
ou pour sauvegarder des résultats. La figure \ref{fig:plot_ex}
montre un exemple d'affichage de résultats et de maillage.

\begin{figure}[h]
    \centering
    \includegraphics[scale = 0.8]{img/plot_ex.png}
    \caption{Exemple d'affichage disponible dans le code}
    \label{fig:plot_ex}
\end{figure}

\newpage

\section{Comparaison des résultats avec la théorie}

Pour comparer notre code avec la théorie,
nous proposon de revenir à la situation initiale
d'une pointe dans une géométrie sphérique.

Nous utiliserons donc le maillage sphérique proposé
par notre code et nous comparerons les résultats
avec la solution analytique.

\subsection{Solution analytique}

Nous avons posé la condition limite suivante:

\begin{equation}
    V_1(\theta) = 1 - \frac{\theta^2}{\theta_0^2}
\end{equation}

En notant $V_1$ la fonction de potentiel en $r=1$.

Au vu de la géométrie, son cherche une décomposition
de la fonction potentiel en coordonnées sphériques
avec des solutions connus de l'équation de Laplace,
c'est-à-dire les harmoniques sphériques (car leurs
laplaciens sont nuls).

Ces fonctions notées $\Phi_n(r, \theta)$ s'écrivent:

\begin{equation}
    \Phi_n(r, \theta) =
    r^{\frac{(2n+1)\pi}{2 \theta_0}}
    cos\left(\frac{(2n+1)\pi}{2\theta_0}\theta\right)
\end{equation}

Ces fonctions sont solutions de l'équation de Laplace
et forment une base de l'espace des fonctions harmoniques
sphériques, tout en vérifiant les propriétés qui
nous intéressent, à savoir qu'elles s'annulent bien
en $\theta = \pm \theta_0$.

Dès lors, il existe une suite $(a_n)$ telle que:

\begin{equation}
    V(r,\theta) = \sum_{n=0}^{+\infty} a_n \Phi_n(r, \theta)
\end{equation}

Pour calculer les coefficients $a_n$, nous allons exploiter
la condition limite en $r=1$.

\begin{equation}
    1 - \frac{\theta^2}{\theta_0^2}
    = \sum_{n=0}^{+\infty} a_n
    cos\left(\frac{(2n+1)\pi}{2\theta_0}\theta\right)
\end{equation}

On reconnait ici la décomposition en série de Fourier
de la fonction $V_1$. On peut donc d'après Fourier exprimer
les coefficients $a_n$ comme suit:

\begin{equation}
    a_n = \frac{1}{\theta_0}
    \int_{-\theta_0}^{\theta_0}
    \left(1 - \frac{\theta^2}{\theta_0^2}\right)
    cos\left(\frac{(2n+1)\pi}{2\theta_0}\theta\right) d\theta
\end{equation}

Nous allons résoudre cette équation pour tout n en réalisant
deux intégrations par partie.

\begin{align*}
    a_n & = \frac{1}{\theta_0}
        \int_{-\theta_0}^{\theta_0}
        \left(1 - \frac{\theta^2}{\theta_0^2}\right)
        cos\left(\frac{(2n+1)\pi}{2\theta_0}\theta\right)
        d\theta \\
        & = \frac{1}{\theta_0}
        \left(
            \left[ \frac{-2\theta_0}{(2n+1)\pi}
             sin\left(\frac{(2n+1)\pi}{2\theta_0}\right)
             \frac{2\theta}{\theta_0^2} \right]_{-\theta_0}^{\theta_0}
             + \int_{-\theta_0}^{\theta_0}
             \frac{2\theta_0}{(2n+1)\pi}
             sin\left(\frac{(2n+1)\pi}{2\theta_0}\right)
             \frac{2\theta}{\theta_0^2}
         \right) \\
        & = \frac{2}{(2n+1)\pi}
        \left( 0 +
            \left[ \frac{-2\theta_0}{(2n+1)\pi}
             cos\left(\frac{(2n+1)\pi}{2\theta_0}\right)
             \frac{2\theta}{\theta_0^2} \right]_{-\theta_0}^{\theta_0}
             + \int_{-\theta_0}^{\theta_0}
             \frac{2\theta_0}{(2n+1)\pi}
             cos\left(\frac{(2n+1)\pi}{2\theta_0}\right)
             \frac{2}{\theta_0^2}
         \right) \\
        & = \frac{2}{\left((2n+1)\pi\right)^2} 
        \left(0 +
            \left[\frac{2\theta_0}{(2n+1)\pi}
            sin\left(\frac{(2n+1)\pi}{2\theta_0}\right)
            \frac{4}{\theta_0}
             \right]_{-\theta_0}^{\theta_0}
         \right) \\
        & = \frac{16}{\left((2n+1)\pi\right)^3}
        \left(
            sin\left(\frac{(2n+1)\pi}{2}\right)
            - sin\left(\frac{-(2n+1)\pi}{2}\right)
        \right) \\
        & = \left(-1\right)^n \frac{32}{
            \left((2n+1)\pi\right)^3}
\end{align*}

Ainsi, on peut écrire la forme générale du potentiel $V$:

\begin{equation}
    V(r, \theta) = \sum_{n=0}^{+\infty}
    \left(-1\right)^n \frac{32}{
        \left((2n+1)\pi\right)^3}
    r^{\frac{(2n+1)\pi}{2 \theta_0}}
    cos\left(\frac{(2n+1)\pi}{2\theta_0}\theta\right)
    \label{eq:analytic}
\end{equation}

Au vu de la forme des coefficients, on peut considérer que 
les dix premiers termes de la série suffisent à obtenir une
bonne approximation de la solution.

\newpage

\subsection{Résultats}
\label{sec:results}

Nous avons donc résolu le problème présenté en section
\ref{sec:situation_physique} avec notre code et comparé
les résultats avec un calcul analytique à l'ordre 10
en utilisant la formule l'équation \ref{eq:analytic}.
Les résultats sont visible figure \ref{fig:analytic_vs_numeric}.

\subsubsection{Comparaison du potentiel}

La figure \ref{fig:analytic_vs_numeric} montre la comparaison
entre les solutions numérique (à gauche) et analytique (à droite).

\begin{figure}[!h]
    \centering
    \includegraphics[scale=0.5]{img/comparison.png}
    \caption{Comparaison des résultats analytiques et numériques}
    \label{fig:analytic_vs_numeric}
\end{figure}

Les formes des potentiels se ressemblent beaucoup.

Pour emphaser les différences, on trace la l'erreur relative
du code par rapport aux valeurs théoriques en chaque point du plan
dans la figure \ref{fig:diff} en utilisant la simple différence
relative $(V_{th} - V_{calc}) / V_{th}$ (en valeur absolue).

\begin{figure}[!h]
    \centering
    \includegraphics[scale=0.5]{img/difference.png}
    \caption{Différence entre les solutions analytiques et numériques}
    \label{fig:diff}
\end{figure}

Ces différences sont surtout visibles aux alentours de la pointe
et le long de celle-ci. Cela s'explique très simplement par
la forme du maillage dans la simulation. En effet, lors de la
définition de la géométrie, le maillage de la pointe n'est pas
contraint de suivre le bord de celle-ci.

On aurait pu ajouter
cette contrainte, mais nous avons préféré garder une construction
de maillage simple et automatique, qui introduit donc des 
discontinuités aux arrêtes de la pointe.

Malgré tout, les différences de potentiels restent satisfaisante
sur l'ensemble du maillage. L'échelle a néanmoins
été forcée à 100\%, des différences plus élevées apparaissent au niveau
de la pointe elle-même, principalement à cause de la géométrie
approximée.

\newpage

\subsubsection{Comparaison du champ électrique}

Nous allons maintenant comparer les champs électriques
calculés par notre code et par la théorie.

Précisons déjà que lors d'une résolution par MEF avec des polynômes
de degré 1, le champ électrique est nécessairement constant
au sein de chaque élément triangulaire. En effet, si chaque
l'équation qui décrit le potentiel dans un élément est de degré 1
par rapport à chaque coordonnée, alors le champ électrique
est de degré 0, c'est-à-dire constant, dans chaque élément.

\newpage

La figure \ref{fig:field_comparison} montre la comparaison
entre les champs électriques
numériques (à gauche) et analytiques (à droite).


\begin{figure}[!h]
    \centering
    \includegraphics[scale=0.5]{img/comparison_elec.png}
    \caption{Comparaison des champs électriques analytiques et numériques}
    \label{fig:field_comparison}
\end{figure}

La figure \ref{fig:field_diff} montre la différence relative entre
les champs électriques numériques et analytiques
$(E_{th} - E_{calc}) / E_{th}$ (en valeur absolue). Une fois
de plus, les différences sont minimes expecté sur la pointe,
principalement à cause de la géométrie du maillage.

On observe également les mêmes discontinuités sur le bord
de la pointe que pour le potentiel.

\begin{figure}[!h]
    \centering
    \includegraphics[scale=0.5]{img/difference_elec.png}
    \caption{Différence relative entre les champs électriques analytiques et numériques}
    \label{fig:field_diff}
\end{figure}

\newpage

\subsubsection{Influence du nombre de nœuds}

Lorsqu'on augmente le nombre de nœuds, on s'attend à ce que les écarts
diminuent. Et c'est bien ce qu'on observe en pratique.

\begin{figure}[!h]
    \centering
    \includegraphics[scale=0.5]{img/difference_elec_7651.png}
    \caption{Illustration de l'impact du nombre de nœuds}
    \label{fig:nodes_number_impact}
\end{figure}

La figure \ref{fig:nodes_number_impact} montre la différence
relative du champ électrique pour deux nombres de nœuds
différents, 7651 et 2791 nœuds. On remarque que les
différences visibles sur l'ensemble du maillage diminuent,
de même que la valeur maximale de la différence, ce qui
signifie que nous nous approchons d'une meilleure solution.

On peut aussi voir que la valeur de champ à la pointe
se rapproche de la valeur calculée (non affiché dans ce document).

Cette tendance devrait bien entendu être étudiée pour un
plus grand nombre d'échantillon pour être validée,
mais compte tenu des performances moyenne du code
en terme de temps de calcul, nous
n'iront pas plus loin dans cette analyse.

\newpage

\section{Conclusion}

D'après les résultats de la section \ref{sec:results}, nous pouvons
dire que le code a un comportement statisfaisant vis-à-vis de la
situation simulée. De plus, il renvoie des valeurs cohérentes de
potentiel et de champ électrique.

Ces résultats peuvent être améliorés, notamment en définissant
une géométrie plus adaptée avec un maillage "lisse" au bord
de la pointe. L'architecture actuelle du code permet en tout
cas une implémentation facilitée de nouvelles géométries.

\printbibliography[title={Références}]

\end{document}